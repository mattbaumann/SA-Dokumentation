\section*{Management summary}

\subsection*{Ausgangslage}
Microservices stellen eine moderne Art der service oriented architecture dar. Die Vorteile der Microservice-Architektur sind unter anderem die Trennung von Aufgabenbereiche in kleinere, verteilte, atomare Applikationen, die Definition der Datenhoheit und die Skalierbarkeit.

Diese Vorteile werden im Konzept der Microfrontends in die Präsentationsschicht übernommen. Um dies zu erreichen muss der \ac{UI}-Code auf die Services aufgeteilt werden und vor der Anzeige zusammengefügt werden.

\subsection*{Vorgehen}
Diese Arbeit untersucht das Zusammenfügen mithilfe von drei Architekturalternativen. Diese werden mithilfe eines Kriterienkatalogs verglichen und bewertet. Es ist nicht das Ziel der Arbeit die beste Alternative zu finden, sondern die Chancen und Risiken der Auswahlmöglichkeiten zu beleuchten und bei der Entscheidungsfindung zu unterstützen.

Der Architekt hat zur Überprüfung der Kriterien eine Prototypimplementierung der Möglichkeiten ausgearbeitet. In der Ergebnissektion wird auf die Erkenntnissen aus der Programmierung der Prototypen verwiesen.

Für die Implementierung der Beispielapplikation wurde Spring\footnote{\url{https://spring.io/}}, Thymeleaf \footnote{\url{https://www.thymeleaf.org/}} und React\footnote{\url{https://reactjs.org/}} im Frontend eingesetzt.

\subsection*{Ergebnisse}
Die Ergebnisse zeigen die Erkenntnisse aus der Implementierung der Alternativen auf. Auch wurden literarische Quellen herbeigezogen bei Kriterien 

Die Arbeit hat die Alternativen nicht gewertet. Es ist in jedem Einzelfall die Gewichtung der Kriterien zu bestimmen um die optimale Implementationsmöglichkeit zu finden.

\subsection*{Ausblick}
Der Verfasser sieht in folgenden Themen noch Möglichkeiten zur Weiterführung der Arbeit.

Eine Untersuchung des \citetitle{newman2015} Patterns von \citeauthor{newman2015}. Es kann dann auch einen Vergleich zu den Alternativen dieser Arbeit gezogen werden.

Eine Weiterführung der Beispielapplikation von dem Prototyp, der in dieser Arbeit erstellt wurde, in eine umfangreichere Version, die die Alternativen genauer einführt.