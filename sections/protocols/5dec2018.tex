\documentclass{article}
 
\begin{document}
    
\subsection{5. Dezember 2018} \label{chap:Protokols:5dec2018}

\subsubsection{Entscheidungen letzten Woche / Arbeiten erledigt}
\begin{itemize}
    \item Die Änderungsvorschläge vom Meeting am 14. November in die Dokumentation nehmen.
    \item Die Kritikpunkte am Kriterien-Katalog überarbeiten
    \item Die Technologiealternativen dokumentieren.\checkmark
\end{itemize}

\subsubsection{Traktanden}
\begin{itemize}
    \item Definieren der Checkliste, wie ich das Dokument einzureichen habe anhand der Checkliste des Studiengangs.
    \item Viele Kriterien die ich aufgestellt habe, können durch die Beispielapplikation nicht überprüft werden. Darf ich diese annehmen oder soll ich solche einfach streichen?
    \item Kritik am technischen Teil (Seite 23-36)
    \item Kritik an Abstrakt (Seite 3)
\end{itemize}
\subsubsection{Protokoll}
\begin{itemize}
    \item Überprüfen der Quallenangaben auf Vollständigkeit
        \begin{itemize}
            \item Autor
            \item URL
            \item Herausgeber
            \item Datum
        \end{itemize}
    \item Die NFA der Applikation sind die Einträge des Kriterien Kataloges
    \item Die Qualitätsattribute sind die ausgefüllen Kriterienvergleiche
     \item Das Repository bleibt privat zwischen Matthias Baumann, Olaf Zimmermann und Mirko Stocker.
     \item Die Bewertung der Arbeit erfolgt gemäss Schema des Studiengangs, siehe Projektwoche 1 und 2, für die Teilnoten 3 bis 5 werden der abgegebene interne Bericht sowie die Beispielanwendung (Ausführung und Source Code) verwendet. Notenrelevante Teile der Dokumentation müssen im Anhang der Arbeit zu finden sein. Für die Notenfindung wird der Bericht und die Ausführung der Beispielanwendung bewertet.
    \item Drei Maturitäten in der Bewertung, (L) Literatur und (I) innerhalb der Implementierung. (M) wie Meinung.
    \item Kapitel 1.2 Streichen, da keine Eigenleistung.
    \item Kritisch hinterfragen ob die Bewertung gemäss Kriterien-Katalog von Technologie, Architektur oder Implementierung begründet sind.
    \item Rückmeldung: Das Projektmanagement der letzten Wochen hat sich verbessert.
    \item Nächste Woche mit Olaf Zimmermann, übernächste Woche mit Mirko Stocker und Olaf Zimmermann.
\end{itemize}
\subsection{Nächste Schritte}
\begin{itemize}
    \item Weiter Dokumentieren
    \item Kritikpunkte korrigieren und bei nächsten Kapitel vermeiden
    \item Einreichen neuer Kapitel vor nächster Sitzung
\end{itemize}
\end{document}
