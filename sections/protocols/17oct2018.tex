\documentclass{article}
 
\begin{document}        
    
\subsection{17. October 2018}

\subsubsection{Entscheidungen letzten Woche / Arbeiten erledigt}
\begin{itemize}
    \item CA commercial gateway order handed in at CA, but product key not already received.
    \item Programmed sample application service with following functions
        \begin{itemize}
            \item Database with H2 running locally with saving data to file
            \item O-R Mapping of the entities with Hibernate. Automatic database schema creating using the entities 
            \item REST API works with spring-boot-data-rest and HATEOAS. That provides Objective-JSON mapping, HATEOAS linking between data etc.
            \item REST API Testing for testing data-management with integration tests. These tests invoke the REST API and prove the requirements are met.
            \item Page rendering with Thymeleaf works with Controller class.
            \item Static page serving, important for skeleton page serving possible
        \end{itemize}
    \item The micro service for kitchen device and purchase list were started and are matured.
\end{itemize}

\subsubsection{Themen in der Besprechung (Planung)}
\begin{itemize}
    \item I reviewed the time I used for different tasks. It feels like I use a long time for project tasks. See Difficulties PDF
    \item I still do not understand backend composition. Therefore, I need to do more research in this topic.
    \item I propose for the criteria catalog, to leave the example as they are, but add the actual criteria for success below. I also try to use the examples for conveying my idea why the criteria is important. So only positive and high quality examples.
    \item Project tells me that my project progress is a week behind the current date. May I rescale the size of the application and how to document the reason and practice of the scope reduction.
\end{itemize}

\subsubsection{Protokoll}
\begin{itemize}
    \item Alle Protokolle und die Dokumentation wird auf Deutsch geschrieben
    \item Umschreiben der Kapitel, die schon geschrieben wurden, und auf Deutsch übersetzen, nicht Protokolle oder Artefakte im Git
    \item Die nächste Version des Kriterien Kataloges, die das Feedback der ersten englischen Version berücksichtigt und wird auf Deutsch verfasst.
    \item Entscheide der letzen Wochen finden sich im Protokoll wieder, mit der Bezeichnung nicht gemacht ode gemacht wenn so.
    \item Sorgfalt, Projektmangement und Kommunikation mit dem Auftraggeber 1. Teilnote, ist bis einschliesslich W4 nicht ausreichend
    \item Massnahme, 1. auf Deutsch schreiben 2. Massnahme, detailierte, gemeinsame Planung und Nachbesprechung der Arbeitspakete oder Projektaktvitäten in den Gemeinsamen Sitzungen.
    \item Aufbau des Protokolles ist wie folgt:
        \begin{enumerate}
            \item Planung
            \item Protokollierung
            \item Nachbesprechung
        \end{enumerate} 
    \item Da im Projekt im Verzug gemäss Projekt Plan, Entscheidung durch seit letztem Meeting, dass aktuell die Beispiel Implementierung, der redaktionellen Arbeit am Kriterien Katalog \& Bericht vorzuziehen war.
    \item Entwürfe der drei Komponenten Model für die drei Architektur Optionen angeschaut. Rückmeldung Betreuer: Sinnvolle Verwendung von Diagrammelementen
    \item Wichtig ist die Spezifikation der Interfaces zwischen den Komponenten
    \item Teilkommenare zu allen Diagrammen: Siehe Hardcopy, Printout, Bsp. in der Gateway alternative, keien restfull HTTP zwischen den Services.
    \item Im begleiteten Text auf Komposition-Diagramm, auf Richrdson Pattern bezugnehmen und die Patterns, die verwendet wurden referenzieren.
    \item Integration-Test des Versuchaufbaus zurückstellen, falls genügend Zeit vorhanden, kann dies noch nachgeholt werden.
\end{itemize}

\subsubsection{Nächste Schritte}
\begin{enumerate}
    \item Grundauftrag, Fragments vorbereiten und Recipe Service abschliessen.
    \item Überarbeiten der Diagramme
        \begin{enumerate}
            \item Komponenten-Diagramme, siehe Hard-Copy
            \item Requirements-Diagramm mit Glossar, das die Farbkodierung für den Leser erklärt
            \item Kopieren der Komponenten-Diagramme für OZ
            \item Cooking Book -> Cook Book Service
        \end{enumerate}
    \item Reflexion auf lessons learned, input für Kriterienkatalog aus Erfahrungen der Programmierung und Modellierung letzten zwei Wochen.
    \item Umschreiben der Kapitel, die schon geschrieben wurden, und auf Deutsch übersetzen, nicht Protokolle oder Artefakte im Git
    \item Ausblick: Security, Policy decision point und Policy enforcement point wohin diese zu setzen. 
\end{enumerate}
\end{document}
