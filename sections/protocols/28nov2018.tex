\documentclass{article}
 
\begin{document}
    
\subsection{28. November 2018} \label{chap:Protokols:28nov2018}

\subsubsection{Entscheidungen letzten Woche / Arbeiten erledigt}
\begin{itemize}
    \item Reflexion auf lessons learned, input für Kriterienkatalog aus Erfahrungen der Programmierung und Modellierung letzen zwei Wochen.
    \item Architekturen vergleichen anhand von qualitativen Merkmalen
    \item Kriterien Katalog auf Excel Tabelle \checkmark
    \item Zeitplan für 50h bis zur Abgabe \checkmark
\end{itemize}

\subsubsection{Traktanden}
\begin{itemize}
    \item Entscheiden wie die Restzeit weiter verwendet werden soll. Dazu kann provisorischer Zeitplan angesehen werden.
    \item Fragen zu Dokumentationsanforderungen:
        \begin{itemize}
            \item Gemeinsames Abnehemen der Applikationen und Kompositionen
            \item Soll ich ein Plakat erstellen für die SA?
            \item Bei der SA-Dokumentations-Anforderungsdokument steht: ``Ergebnisse der Arbeit: was wurde erreicht, was wurde nicht erreicht, Ursachen''. Dies überschneidet sich mit unserer Entscheidung den CA Gateway nicht zu dokumentieren. Ist es korrekt, wenn ich den CA API Gateway weiterhin nicht dokumentiere?
            \item Die Anforderungen für die Dokumentation sind sehr auf Entwicklungsarbeiten ausgelegt. SE-Artefakte werden dort in der Dokumentation gefordert. Ich werde diese nur im Repository dokumentieren.
        \end{itemize}
        \item Ich wollte den Vorschlag von Mirko Stocker umsetzen und Swagger ins Projekt einfügen. Leider ist die Bibliothek ``springfox-swagger2'' nicht anwendbar bei meinem Projekt. Gibt es eine Alternative um eine API Doku im Service generieren kann?
        \item Kriterien Katalog zum Review gegeben. Besprechung des Reviews.
        \item Auf nächste Woche: Ausarbeitung der Vor- und Nachteilen der Architekturformen.
\end{itemize}
\subsubsection{Protokoll}
\begin{itemize}
    \item Poster mit maximal 1h vorbereiten. Dies ist als Übung gedacht und wird nicht massgeblich bewertet.
    \item Abstrakt bis 12. Dezember 11:00 schreiben und abgeben um einen Review zu haben vor der Abgabe.
    \item Abgabe auf einer Datendisk (CD, DVD) die die öffentlichen und privaten Versionen der Dokumentation und Source Code enthält.
    \item Die drei Varianten beschreiben wie sie in der Ist-Form programmiert worden sind. Es kann darüber hinaus einen Ausblick geben, bei welchem die weiteren Anknüpfungspunkte angegeben werden können.
    \item Zu dokumentierende Kompositionen: Frontend Data Komposition, Backend Komposition und Portal Komposition.
    \item Die Software Engineering Artefakte die für das Forschungsziel wichtig sind, können in den Posa kommen. Weitere Artefakte können in den öffentlichen Anhang genommen werden, müssen dann den redaktionellen Qualitäten des Gesammtberichtes entsprechen.
    \item Das Thema Swagger in die Fussnote nehemen um dem Leser die Problematik mit dem Spring-Fox näher zu bringen und die Entscheidung, den Text in Posa zu machen, zu vermitteln. Mit Link auf Issue.
    \item Kriterien Katalog, SMART in Kriterien nehemen, z. B. Landing Zones benutzen um die gut und schlecht klarer abgrenzen.
    \item Nach Absprache sollte Aufgabenstellung in den Anhang nehmen. Die Checkliste ist normativ und diese Inhalte müssen in der Dokumentation vorkommen.
    \item Die Theoretische Skalierbarkeit überarbeiten, Suggestiv-Sätze durch neutrale austauschen.
    \item Microfrontends Patterns ist ein verwirrender Begriff, besser verwenden: Technologietrend, Architekturparadigma, Modularisierungsansatz
    \item Vorschlag für den Ausblick, ein nächster Logischer Schritt, Patterns zu schreiben aus den 
\end{itemize}
\subsection{Nächste Schritte}
\begin{itemize}
    \item Die Änderungsvorschläge vom Meeting am 14. November in die Dokumentation nehmen.
    \item Die Kritikpunkte am Kriterien-Katalog überarbeiten
    \item Die Technologiealternativen dokumentieren.
\end{itemize}
\end{document}
