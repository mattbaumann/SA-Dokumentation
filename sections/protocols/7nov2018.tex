\documentclass{article}
 
\begin{document}
    
\subsection{7. November 2018} \label{chap:Protokols:7nov2018}

\subsubsection{Entscheidungen letzten Woche / Arbeiten erledigt}
\begin{itemize}
    \item Reflexion auf lessons learned, input für Kriterienkatalog aus Erfahrungen der Programmierung und Modellierung letzen zwei Wochen.
    \item Kriterien Katalog auf Excel Tabelle
    \item Architekturen vergleichen anhand von qualitativen Merkmalen 
    \item Überarbeitung der Beispielservices - nicht ausgeführt aus Zeitmangel
    \item Composition Diagram umdrehen
    \item Tabellen, Webdesign, URL (URI Namensraum detailliert beschreiben) \checkmark
    \item Client Server Cuts und Client / Server Layering in der Vorgeschlagenen IRP (informal rich picture) darstellen und beschreiben. \checkmark
    \item Portal Integration läuft stabil und besitzt alle Funktionen \checkmark
    \item Backend Integration läuft stabil ist noch nicht vollständig ausprogrammiert \checkmark
\end{itemize}

\subsubsection{Traktanden}
\begin{itemize}
    \item Nächste Woche 13:00 im Raum 1.275
    \item URL Namensraum ist vorbereitet. Dies erforderte viel lesen, verstehen von Spring, und standardisieren von URLs im Prototyp
    \item Wegen oben habe ich den Kopf voll mit technischen Details von Spring gehabt. Aus diesem Grund habe ich diese Woche Prototyp Integration programmiert.
    \item Portal Integration
    \item Backend Integration
    \item Ich habe keine weiteren Features dem Framework hinzugefügt, da ich dies tief priorisiert habe.
    \begin{itemize}
        \item erhöht den Arbeitsaufwand bei den Integrations
        \item Nicht wichtig für Kernaufgabenstellung, Niedrige Priorität
    \end{itemize}
    \item informal rich picture 
    \item Nächste Woche: Dokumentation nachholen an 1.5Tage und 0.5 Tage um Überzeitabbau
\end{itemize}

\subsubsection{Protokoll}
\begin{itemize}
    \item ``Backend for Frontend'' von Sam Newman auf Relevanz prüfen
    \item Kriterien für Katalog
        \begin{enumerate}
            \item Performance
            \begin{itemize}
                \item Seitenladezeit nicht zu lange
                \item Speicherverbrauch
                \item Grösse der Website in der Übertragung sollte unter 5 Megabyte verbleiben
            \end{itemize}
            \item Zusammenstellung von verschiedenen Komponenten muss im Design aufeinander abgestimmt sein
            \item  Kommunikation zwischen den Komponenen (Microfrontends)
            \item  Robustheit durch Isolation der Komponenten. Beispielsweise darf Komponenten nicht ausfallen weil sie keine Antwort von Peer bekommt
            \item Security
                \begin{itemize}
                    \item SSL und TLS muss unterstützt werden
                    \item SSL und TLS Endpoints dürfen sich nicht in die Quere kommen.
                    \item Keine Seiteneffekte bei verschiedenen SSL Zertifikaten
                    \item CIA
                    \item Authentication, keine zwei Logins für Nutzer
                \end{itemize}
        \end{enumerate}
    \item Bewusste Entscheidung von Browsern für Testen des Frontends mit Firefox und Chrome. Optional noch Microsoft Edge. 
    \item Lighthouse Chrome Plugin gibt eine Qualitätsscore für Mobileanwendungen.
    \item Genaue Dokumentierung von Ausgangssituationen
    \item Ansehen des Kriterienkataloges mit Mirko Stocker
    \item Kriterien für Katalog
        \begin{enumerate}
            \item SEO, Search Engine Optimization, gutes Markup ohne Javascript
            \item Ad Blocker sollen nicht die Funktionalität der Site beinträchtigen.
            \item Umgang mit Cookies, sowohl selber ausgestellte und Tracing Cookies von 3. Seite
            \item Umgang mit Ad Blocker muss gewährleistet werden
            \item E-Tag und cache invalidierung muss beim Update eines Services korrekt funktionieren
        \end{enumerate}
    \item Mit Facebook Debugger kann das SEO von der Seite getestet werden. Facebook Debugger nimmt URL und gibt Informationen des Crawling zum Entwickler zurück.
    \item Sitzung am 14.11.2018 mit Olaf Zimmermann und 21.11.2018 mit Mirko Stocker. Es ist an der gleichen Zeit und Ort bei beiden Terminen.
    \item IFS-Logo aktualisieren auf aktuelle Version.
\end{itemize}
\end{document}
