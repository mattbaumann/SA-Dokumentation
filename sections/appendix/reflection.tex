\section*{Persönliche Reflexion}

Ich \marg{Beweggrund} und mein SA-Partner haben uns für diese SA-Aufgabenstellung entschieden weil wir davon ausgegangen sind, dass die Aufgabenstellung optimal auf unsere Fähigkeiten passt. Ich wollte die Forschungsarbeit übernehmen und mein Partner war sehr interessiert an der praktischen Umsetzung der theoretischen Ergebnissen.

Wir \marg{Lerngegenstände} haben eben diese Arbeit gewählt, weil mein Partner ein tiefes Verständnis für die angewendeten Bibliotheken hatte. Nachdem mein Partner aus der Zusammenarbeit ausgeschieden ist, musste ich dieses Wissen im Alleingang selber aneignen. Zum Beispiel habe ich folgende Bibliotheken durch Recherche selber kennen gelernt:

\begin{description}
    \item[Spring] Spring Boot, Spring Framework, Spring Data, Spring Data Rest, Spring WebMVC, Spring WebFlux (WebClient), JUnit 5, RestAssured
    \item[Thymeleaf] Die Thymeleaf Server-Rendering Sprache
    \item[React] React JS, React Starter Kit, React Semantic UI, Angular JS
\end{description}

Ausserdem konnte ich während meiner SA folgende Programme und Werkzeuge kennen lernen:
\begin{description}
    \item[Enterprise Architect] Ein UML-Zeichnungswerkzeug
    \item[Microsoft Project] Ein Projektplaner
    \item[Microsoft SharePoint Server] Verwaltung von Dokumenten mit Versionierung
    \item[Eclipse IDE] Eine Java-Entwicklungsumgebung
    \item[Gradle] Ein Build-Tool für Java
    \item[Latex] Ein Dokumentenlayouter
    \item[HTTP Files] Ein Fileformat das verschiedene HTTP Headers speichert, die durch ein dafür ausgelegtes Programm (IntelliJ, Visual Studio Code) an den Server gesendet werden um einen manuellen Integrationstest zu ermöglichen
\end{description}

Mein SA-Partner \marg{Risiken} hat nicht alle notwendigen Prüfungen bestanden und ist damit in der ersten Woche weggefallen. Dieser Verlust wird sich im Verlauf der SA als grosse Behinderung erweisen. Zum Anfang des Projektes hatte ich keinen Zuhörer mit welchem ich die Problemstellungen der Architektur durchgehen konnte. Statt einer kooperativen Problemanalyse der Architektur, habe ich selber versuchen müssen die Chancen und Risiken durch aktives Testen, Implementieren und Verifizieren auf den Grund zu gehen. Dies führte zu Verzögerungen beim Projektvortschritt.

Die Einarbeitung hat sehr viel Zeit gekostet. Im Speziellen hat die Einarbeitung in die Bibliotheken hat sehr viel Zeit und Mühe gekostet. Immerhin hatte ich es mir nicht ausgesucht, dass ich diese Arbeit alleine schreiben muss und mir diese Grundlagen fehlen. Ausserdem war es nicht sehr hilfreich, dass die meisten Themen im Spring Framework durchschnittlich zwei Wochen nach der Fertigstellung der Komponente im Unterricht von AppArch besprochen wurden. 

Aus diesem Grund hatte ich sehr viel Zeit in die Einarbeitung von nicht zielrelevanten Informationen stecken müssen, da in den Meetings vereinbart wurde, dass die Applikation einen Auslieferungsgegenstand der Arbeit sei. Dies führte zu weiteren Verzögerungen, die es mir nicht ermöglichten alle Themen der SA zu bearbeiten. Im Speziellen hätte die Ausarbeitung von Kriterien im Kriterienkatalog Versuche mit der Beispielsanwendung gebraucht für welche keine offene Zeit mehr übrig blieb.

In der Retrospektive stellte sich heraus, dass viele Entscheidungen in den Sitzungen zu nicht optimalen Lösungen geführt hatten. Zum Beispiel sei da zu erwähnen, dass das Weglassen der Beispielanwendung eine viel sinnvollere Einengung der Arbeitsstunden gegeben hätte als die punktweise Einengung, die von meinem Betreuer vorgeschlagen wurde.


% Vor dem Schreiben der Retrospektive hatte ich das Thema Zeitplan dann mit anderen Dozenten der HSR besprochen und es stellte sich heraus, dass das Erarbeiten eines Zeitplanes in der ersten Woche eine typische ''Wasserfall`` Arbeit ist und definitiv nicht geeignet für eine Explorationsaufgabe. Im Fach SE1 und SE2  wird die Scrum-Methode vorgeschlagen. Bei einer solchen Arbeitsmethode wird auch ein Zeitplan erstellt, aber die Arbeiten werden von Meeting zu Meeting neu definiert. Diese Methode erlaubt das Eingehen auf Schwierigkeiten in der Arbeit und zwingt den Fahrplan nicht auf.

Zu meinem Erstaunen, wurde in der Sitzung vom 5. Dezember der Code von der Bewertung ausgeschlossen und nur die Dokumentation hereingenommen. Dies wurde am Anfang des Projektes anders kommuniziert und hätte mich dazu veranlasst die Beispielanwendung von Anfang an wegzulassen.

Als \marg{Chancen} positive Konsequenz nahm ich ein tiefes Verständnis für Web-Technologien mit auf den Weg in mein Arbeitsleben. 

Zusammen mit dem Fach PmQm konnte ich meine ersten Erfahrungen in der Projektgestaltung machen. Ich nehme mit, dass ich mich verschätzt habe in der Einteilung der Vorstudie des Projektes. In der BA werde ich mehr Zeit einteilen für die Untersuchung von vorhandener Dokumentation. Mit dem Studium der Dokumentation wären Projektrisiken, als Beispiel sei der API Gateway genannt, früher zu beseitigen gewesen.

Ich arbeitete zu Beginn des Projektes einen Zeitplan aus, nach dem ich die Arbeit einteilte. Der Vorschlag ''Wasserfall``-Projektplan verwarf ich wegen der Komplexität der Aufgabenstellung. Ich habe stattdessen mit Erfolg eine agile Arbeitseinteilung einsetzen können. Diese Arbeitsmethode erlaubt den schrittweisen Erfahrungsaufbau und die Verbesserung durch mehrmaliges Überarbeiten der Erkenntnisse und Experimente. Dies hatte den Vorteil, dass ich richtungweisende Entscheidungen schnell treffen konnte. Nachteilig war, dass wichtige Entscheidungen (siehe API Gateway) erst spät im Projekt und kurzfristig getroffen wurde, das Projekt hat somit vereinzelt starke Änderungen der Anforderungen erfahren müssen.

Wie im Projektplan angegeben habe ich zuerst experimentiert und die Ergebnisse aufgeschrieben. Bei der BA werde ich versuchen die Ergebnisse während der Experimente in den Text zu schreiben. Ich erhoffe mir, dass das Dokumentieren der Ergebnisse zu weiteren Erkenntnissen führen kann für die Experimente.

Die Stakeholder-Kommunikation konnte ich während des Projektes verbessern. Die Meetings sind besser strukturiert gewesen, was auch daran gelegen hat, dass ich ein tieferes Verständnis für Materie entwickelte.

Im Endeffekt \marg{Schlussfolgerung} bin ich mit dem Projekt und der Herausforderung gewachsen. Ich werde von dem tiefen Verständnis für Webtechnologien profitieren bei meiner zukünftigen Arbeit, das ich über die SA aufbauen konnte. Und mit der Erfahrung im Projektmanagement erwarte ich, dass ich die BA zu Beginn besser strukturieren kann.
