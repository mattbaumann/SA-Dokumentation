\section{Eignung der Microservice Patterns von Richardson}

Die Microservice Patterns von Richardson sind weitesgehend unausgearbeitet.\cite{microservicesio} Aus den Titel der Patterns konnte der Architekt die Intention des Textes herauslesen. Ohne den Text ist es nicht möglich auf die Details der Patterns zu referenzieren. Aus diesem Grund wird im weiteren Verlauf des Textes darauf verzichtet auf Einzelheiten der Patterns zu referenzieren und davon ausgegangen, dass der Inhalt der Patterns die Interpretation des Autors darstellt.

Der Architekt nutzte die folgenden Patterns für die Arbeit
\begin{itemize}
    \item \citetitle{RichardsonFrontend}\cite{RichardsonFrontend}
    \item \citetitle{RichardsonFragment}\cite{RichardsonFragment}
\end{itemize}

Es ist dem Architekten nicht bekannt weshalb \citeauthor{RichardsonFrontend} die Patterns unstimmig benannt hat. Der Architekt versteht unter den Pattern die Zusammenführung der Daten von verschiedenen Services. Der Architekt versteht den Unterschied so, dass die Datenzusammenführung beim einen Pattern auf dem Client, dem Webbrowser des Users, und beim zweiten auf einem Kompositionsserver ausgeführt wird.

Nach Projektplan sollte auch das \citetitle{RichardsonAPIGateway}\cite{RichardsonAPIGateway} untersucht werden, dieses wurde im Verlauf des Projektes verworfen wegen dem Verwurf der Implementation des Patterns.

In einer Folgearbeit kann das Pattern \citetitle{RichardsonBFF}\cite{RichardsonBFF} weiter vertieft werden, das eine Komposition aufgrund der Anforderungen des Frontends ermöglicht.\cite{newman2015}