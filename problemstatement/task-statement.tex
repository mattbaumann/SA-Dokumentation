 % !TEX root ../main.tex
 
\section{task statement}

\subsection{Industriepartner und Betreuer}
Diese Studienarbeit wird nicht in Zusammenarbeit mit einem externen Partner durchgeführt; im Projektverlauf werden aber ggfs. externe Subject Matter Experts (SMEs) Anforderungen einbringen und Zwischenfeedback zu den Resultaten geben.   
Betreuer HSR: 
Prof. Dr. Olaf Zimmermann, Institut für Software, ozimmerm@hsr.ch  
\subsection{Ausgangslage}
Microservices stellen einen aktuellen Software Engineering (SE)-Trend dar. Microservices lassen sich als hochmodulare, verteilte, teilautonome Komponenten charakterisieren, die viele feingranulare Web und Messaging Application Programming Interfaces (APIs)  exponieren. 
Es stellt sich die Frage, wie aus diesen meist kleinen Servicekomponenten ganze (Web-) Anwendungen aufgebaut werden können, also wo wie komponiert wird: 
Browser (mit verschiedenen Detailoptionen)? 
API Gateway (Beispiel: \url{https://github.com/CAAPIM/Microgateway})? 
Andere API Client Patterns?
Weiterhin ist nicht offensichtlich, was sind die Vor- und Nachteile dieser Lösungsoptionen sind.
\subsection{Ziele der Arbeit und Liefergegenstände}
Bei diesem Projekt handelt es sich um ein exploratives Vorhaben, das dem Wissens- und Erfahrungsaufbau für Lehre und Forschung dient. 
Noch festzulegende Architektur- und Technologiealternativen sollen anhand eines zu erarbeitenden Kriterienkataloges verglichen und bewertet werden. Beispiele für derartige Kriterien sind Einarbeitungsaufwand,  CIA-Sicherheitsanforderungen, Skalierbarkeit, Performance und Wartbarkeit. 
Eine prototypische Umsetzung der zu vergleichenden Alternativen (als prototypische Library bzw. Werkzeug) z.B. zur Verwendung im Übungsbetrieb (VL Application Architecture) soll ebenfalls erfolgen. Dabei sollen noch festzulegende Patterns zu Einsatz kommen wie z.B. die auf dieser Webseite https://microservices.io/patterns/index.html dokumentierten.  
Die folgenden Quellen sollen zur Identifikation der  Architektur- und Technologiealternativen dienen: 
\begin{enumerate}
    \item \href{https://medium.com/@tomsoderlund/micro-frontends-a-microservice-approach-to-front-end-web-development-f325ebdadc16}{Tom Soderlund's Medium post}\footnote{https://medium.com/@tomsoderlund/micro-frontends-a-microservice-approach-to-front-end-web-development-f325ebdadc16}
    \item \href{https://micro-frontends.org/}{micro frontend manifesto}\footnote{https://micro-frontends.org/}
    \item \href{https://ieeexplore.ieee.org/document/7888407/?arnumber=7888407}{Microservices in Practice, Part 1 and Part 2}\footnote{https://ieeexplore.ieee.org/document/7888407/?arnumber=7888407}
\end{enumerate}

\subsection{Unterstützung}
Die erwartete und effektiv erhaltene Unterstützung wird durch den/die Studenten in Sitzungsprotokollen definiert und im SA-Bericht dokumentiert.

\subsection{Zur Durchführung}
Mit dem HSR-Betreuer finden in der Regel wöchentlich Besprechungen statt. Zusätzliche Besprechungen sind nach Bedarf zu veranlassen. 
Alle Besprechungen, bei denen eine Vorbereitung durch den Betreuer nötig ist, sind von dem/den Studenten mit einer Traktandenliste vorzubereiten. Beschlüsse sind in einem Protokoll zu dokumentieren.
Für die Durchführung der Arbeit ist ein Projektplan zu erstellen. Dabei ist auf einen kontinuierlichen und sichtbaren Arbeitsfortschritt zu achten. Arbeitszeiten sind zu dokumentieren. 
Die Spezifikation der Anforderungen geschieht durch den/die Studenten in Absprache mit dem Betreuer. Bei Disputen entscheidet der Betreuer in Rücksprache mit dem/den Studenten über die definitiv für die Studienarbeit relevanten Anforderungen.  
Vorstudie, Anforderungsdokumentation und Architekturdokumentation sollten im Laufe des Projektes mittels Milestone mit dem Betreuer in einem stabilen Zustand abgenommen werden. Zu den abgegebenen Arbeitsresultaten wird ein vorläufiges Feedback abgegeben. Eine definitive Beurteilung erfolgt auf Grund der am Abgabetermin abgelieferten Dokumentation.
Das aus der Arbeit resultierende Architekturwissen soll im SA-Bericht öffentlich verfügbar sein, die Programmierbeispiele ggfs. ebenfalls in einem öffentlichen Open Source Repository zugänglich gemacht werden. 
\subsection{Dokumentation}
Über diese Arbeit ist eine Dokumentation gemäss den Richtlinien der Abteilung Informatik zu verfassen. Die zu erstellenden Dokumente sind im Projektplan festzuhalten. Alle Dokumente sind nachzuführen, d.h. sie sollten den Stand der Arbeit bei der Abgabe in konsistenter Form dokumentieren. Die Dokumentation ist vollständig auf CD/DVD in zwei Exemplaren abzugeben. Bei der Projektdokumentation sind die Anleitungen des Studienganges inklusive Anhängen zu beachten. Zudem ist eine kurze Projektresultatdokumentation für die Webseiten von Prof. Zimmermann erwünscht (ggfs. kurzes Video).

\subsection{Termine} 
Siehe HSR-Webseiten.

Beginn der Studienarbeit, 
Ausgabe der Aufgabenstellung durch die Betreuer

Abgabe A0-Poster

Abgabe Kurzfassung Studienarbeit für Betreuer

Kurzfassungs-Überarbeitung an Studiengangssekretariat

Abgabe Bericht

Allfällige weitere Termine sind am Sekretariat der Abteilung Informatik zu erfragen und sollten entsprechend in einem Sitzungsprotokoll dokumentiert werden. 
\subsection{Beurteilung}
Eine erfolgreiche Studienarbeit zählt 8 ECTS-Punkte pro Studierenden. Für 1 ECTS-Punkt ist eine Arbeitsleistung von ca. 25 bis 30 Stunden budgetiert. 
Siehe \url{http://studien.hsr.ch/allModules/23498_M_SAI.html} für die Modulbeschreibung der Studienarbeiten.
Für die Beurteilung ist der HSR-Betreuer verantwortlich unter Einbezug des Feedbacks der Projektpartner.

\begin{table}
    \begin{tabularx}{\textwidth}{|c|X|}
        \hline 
        Gewicht & Gesichtpunkt \\
        \hline 
        \( \frac{1}{5} \) & Organisation, Durchführung \\ 
        \hline 
        \( \frac{1}{5} \) &  Berichte (Abstract, Management Summary, technische u. persönliche Berichte) sowie Gliederung, Darstellung und Sprache der gesamten Dokumentation. \\ 
        \hline 
        \( \frac{3}{5} \) & \textbf{Inhalt}: Die Unterteilung und Gewichtung von 3. Inhalt wird im Laufe dieser Arbeit festgelegt.
        Im Übrigen gelten die Bestimmungen der Abt. Informatik zur Durchführung von Studienarbeiten.\\
        \hline 
    \end{tabularx}
    \caption{Wighting of the SA work items}
\end{table}

Rapperswil, den 19.09.2018

Prof. Dr. Olaf Zimmermann
Institut für Software
Hochschule für Technik Rapperswil