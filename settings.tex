%%%%%%%%%%%%%%%%%%%%%%%
%% KOMA Script
%%%%%%%%%%%%%%%%%%%%%%%

\KOMAoptions{parskip = half} % full for one line % full for one line
\KOMAoptions{fontsize=12pt}

\recalctypearea % Needed after setting font-size

%%%%%%%%%%%%%%%%%%%%%%%
%% PDF Meta Data
%%%%%%%%%%%%%%%%%%%%%%%

\makeatletter

\hypersetup{xetex,
    pdftitle={\@title},
    pdfsubject={\@subject},
    pdfauthor={\@author},
    pdfkeywords={}
    pdfproducer={\KOMAScript},
    pdfcreator={xetex},
    % pdftoolbar=true, % show Acrobat’s toolbar?
    % pdfmenubar=true,        % show Acrobat’s menu?
    % pdffitwindow=false,     % window fit to page when opened
    % pdfstartview={FitH},    % fits the width of the page to the window
    % pdfnewwindow=true,      % links in new window
    colorlinks=true,       % false: boxed links; true: colored links
    linkcolor=red,          % color of internal links (change box color with linkbordercolor)
    citecolor=green,        % color of links to bibliography
    filecolor=magenta,      % color of file links
    urlcolor=cyan           % color of external links
}

\makeatother

%%%%%%%%%%%%%%%%%%%%%%%%
%% Header and Footer %%
%%%%%%%%%%%%%%%%%%%%%%%%

% \renewcommand{\sectionmark}[1]{\markright{#1}} %entfernt nummer vor section
% \renewcommand{\subsectionmark}[1]{}

\pagestyle{scrheadings}

\makeatletter

% e - even or o - odd
% l - left, c - center, r - right
\newcommand{\headerFooterContent} {

    \clearscrheadfoot

    \KOMAoptions{headsepline = yes}
    \KOMAoptions{footsepline = yes}

    \rohead{\LogoHSR}
    \rehead{\LogoCompany}
    \lohead{\LogoCompany}
    \lehead{\LogoHSR}

    \rofoot{\thepage\ (\theCurrentPage) of \lastpageref*{LastPages}}
    \refoot{\today}
    \cefoot{{ \small \@author}}
    \lefoot{\thepage\ (\theCurrentPage) of \lastpageref*{LastPages}}
    \lofoot{\today}
    \cofoot{{ \small \@author}}

    \recalctypearea % Needed to recalculate all lengths
}

\newcommand{\headerFooterAppendix} {
    \clearscrheadfoot

    \KOMAoptions{headsepline = no}
    \KOMAoptions{footsepline = no}

    % \rohead{\LogoHSR}
    % \rehead{\LogoCompany}
    % \lohead{\LogoCompany}
    % \lehead{\LogoHSR}

    % \rofoot{\thepage\ (\theCurrentPage) of \lastpageref*{LastPages}}
    % \refoot{\today}
    % \cefoot{{ \small \@author}}
    % \lefoot{\thepage\ (\theCurrentPage) of \lastpageref*{LastPages}}
    % \lofoot{\today}
    
    % \recalctypearea % Needed to recalculate all lengths
}

\makeatother

%%%%%%%%%%%%%%%%%%%%%
%% Remove TOF, TOT Titles
%%%%%%%%%%%%%%%%%%%%%
\addto\captionsenglish{
   \renewcommand\listfigurename{}
}

\addto\captionsenglish{
    \renewcommand\listtablename{}
}

% \setkomafont{pageheadfoot}{\small}

%%%%%%%%%%%%%%%%%%%%%
%% Title
%%%%%%%%%%%%%%%%%%%%%

\RedeclareSectionCommand[
    beforeskip=-\baselineskip,
    afterskip=.5\baselineskip,
    indent=1pt]{section}

\RedeclareSectionCommand[
    beforeskip=-\baselineskip,
    afterskip=.5\baselineskip,
    indent=1pt]{subsection}

\RedeclareSectionCommand[
    beforeskip=-\baselineskip,
    afterskip=.5\baselineskip,
    indent=1pt]{subsubsection}

\RedeclareSectionCommand[
    beforeskip=-\baselineskip,
    afterskip=.5\baselineskip,
    indent=1pt]{paragraph}

%%%%%%%%%%%%%%%%%%%%%
%% Index
%%%%%%%%%%%%%%%%%%%%%
\usepackage{imakeidx}
\makeindex[intoc,columnseprule]
\indexsetup{firstpagestyle=plain}    % Show header/footer on index page


%%%%%%%%%%%%%%%%%%%%%
%% Marginals %%
%%%%%%%%%%%%%%%%%%%%%
\newcommand{\marg}[1]{\marginpar{\raggedright \textbf{#1} }}

%%%%%%%%%%%%%
%% Tabular %%
%%%%%%%%%%%%%
\newcolumntype{L}[1]{>{\raggedright\arraybackslash}p{#1}} % Tabelleninhalt linksausgerichtet
\newcolumntype{R}[1]{>{\raggedleft\arraybackslash}p{#1}} % Tabelleninhalt rechtsausgerichtet
\newcolumntype{C}[1]{>{\centering\arraybackslash}p{#1}} %  Tabelleninhalt zentriert


%%%%%%%%%%%%%
%%   Misc  %%
%%%%%%%%%%%%%

\newcommand{\matlab}[1]{\footnotesize{(Matlab: \texttt{#1})}\normalsize{}}

\newcommand\tabbild[2][]{%
	\raisebox{0pt}[\dimexpr\totalheight+\dp\strutbox\relax][\dp\strutbox]{%
		\includegraphics[#1]{#2}%
	}%
}

% Makro für Vorteile und Nachteil mit Plus und Minus
\newcommand\pro{\item[$+$]}
\newcommand\con{\item[$-$]}

\makeatletter

%%%%%%%%%%%%%%%%%%%%%%%%%%%%
% Mathematical Operators %
%%%%%%%%%%%%%%%%%%%%%%%%%%%%
\DeclareMathOperator{\sinc}{sinc}
\DeclareMathOperator{\sgn}{sgn}
\DeclareMathOperator{\Real}{Re}
\DeclareMathOperator{\Imag}{Im}
\DeclareMathOperator{\euler}{e}
\DeclareMathOperator{\cov}{cov}
\DeclareMathOperator{\PolyGrad}{PolyGrad}
\DeclareMathOperator{\gradient}{grad}
\DeclareMathOperator{\rotation}{rot}
\DeclareMathOperator{\divergenz}{div}
\DeclareMathOperator{\imag}{j}

%Grösse Integral anpassen
\def\Int{\mbox{\Large$\displaystyle\int$\normalsize}}
\def\Int{\mbox{\Large$\displaystyle\iint$\normalsize}}
\def\OInt{\mbox{\Large$\displaystyle\oint$\normalsize}}

%Makro für 'd' von Integral- und Differentialgleichungen 
\newcommand*{\diff}{\mathop{}\!\mathrm{d}}

%%%%%%%%%%%%%%%%%%%%%%%%%%%
% Fouriertransform %
%%%%%%%%%%%%%%%%%%%%%%%%%%%

\unitlength1cm
\newcommand{\FT}
{
	\begin{picture}(1,0.5)
	\put(0.2,0.1){\circle{0.14}}\put(0.27,0.1){\line(1,0){0.5}}\put(0.77,0.1){\circle*{0.14}}
	\end{picture}
}


\newcommand{\IFT}
{
	\begin{picture}(1,0.5)
	\put(0.2,0.1){\circle*{0.14}}\put(0.27,0.1){\line(1,0){0.45}}\put(0.77,0.1){\circle{0.14}}
	\end{picture}
}

\makeatother % General barrier